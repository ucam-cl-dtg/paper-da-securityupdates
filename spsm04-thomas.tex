\documentclass{sig-alternate-2013}

\usepackage[utf8]{inputenc}
\usepackage{amsmath}
\usepackage{amsfonts}
\usepackage{amssymb}
\usepackage{graphicx}
\usepackage[space]{grffile}
\graphicspath{figures/}
\usepackage{import}
\usepackage{url}
\usepackage[hidelinks]{hyperref}

% http://tex.stackexchange.com/questions/57865/how-to-use-multiple-urls-for-one-bibtex-reference
\let\URL\url
\makeatletter
\def\url#1{\@URL#1||\@nil}
\def\@URL#1|#2|#3\@nil{%
  \URL{#1}\ifx\relax#2\relax\else\ -- \URL{#2}\fi}
\makeatother

% To stop problems with links in footnotes spanning columns or pages.
\interfootnotelinepenalty=10000

\usepackage{subcaption}
\usepackage{tikz}
\usepackage{siunitx}
\usepackage[disable]{todonotes}%
\let\OldTodo\todo
\renewcommand{\todo}{\OldTodo[inline]}
\newcommand{\todolater}[1]{}%{\todo{#1}}%
\RequirePackage[date=terse, isbn=true, doi=false, url=false, urldate=iso8601, maxbibnames=9, backref=false, backend=bibtex, style=ieee]{biblatex}
\addbibresource{securityupdates.bib}
\renewcommand{\bibfont}{\small}

\AtEveryBibitem{% Clean up the bibtex rather than editing it
 \clearname{editor} % remove editors
}

\newcommand{\daNumSigModels}{18}
\newcommand{\daSigNumDevices}{100}
\newcommand{\daSigNumDeviceDays}{10\,000}
\newcommand{\daSigNumDays}{100}
\newcommand{\daSigNumDevicesDay}{20}
\newcommand{\daSigVersionPerc}{1\%}
\newcommand{\daSigVersionDays}{10}
\newcommand{\daNumVulnsUsed}{11}
\newcommand{\daNumDeviceDataDevices}{50}
\newcommand{\daOSYearsOfData}{4}
\newcommand{\daOSMonthsOfData}{51}
\newcommand{\daStartDate}{2011-07-01}
\newcommand{\daEndDate}{2015-09-22}
\newcommand{\daFullDeployedAt}{95\%}
\newcommand{\daNumManufacturers}{302}
\newcommand{\daNumSigManufacturers}{10}
\newcommand{\daNumModels}{2\,770}
\newcommand{\daNumOperators}{1\,470}
\newcommand{\daNumSigOperators}{14}
\newcommand{\daNumOSDevices}{20\,500}
\newcommand{\daOSVersionPercValidLines}{99.9\%}
\newcommand{\daOSTotalDaysData}{1\,350\,000}
\newcommand{\daMeanInsecurityPercNominal}{87.3\%}
\newcommand{\daMeanInsecurityPerc}{$87.3 \pm 0.0\%$}
\newcommand{\daMeanInsecurityPercTwosfNominal}{87\%}
\newcommand{\daMeanInsecurityPercTwosf}{$87 \pm 0\%$}
\newcommand{\daMeanSecurityPercNominal}{12.7\%}
\newcommand{\daMeanSecurityPerc}{$12.7 \pm 0.0\%$}
\newcommand{\daMeanSecurityPercTwosfNominal}{13\%}
\newcommand{\daMeanSecurityPercTwosf}{$13 \pm 0\%$}
\newcommand{\daVulnFreeNominal}{0.127}
\newcommand{\daVulnFree}{$0.127 \pm 0.0$}
\newcommand{\daMeanOutstandingVulnerabilities}{$0.509 \pm 0.0$}
\newcommand{\daUpdatednessNominal}{0.0553}
\newcommand{\daUpdatedness}{$0.0553 \pm 0.0$}
\newcommand{\daUpdatednessPercNominal}{5.53\%}
\newcommand{\daUpdatednessPerc}{$5.53 \pm 0.0\%$}
\newcommand{\daUpdatednessPercTwosfNominal}{5.5\%}
\newcommand{\daUpdatednessPercTwosf}{$5.5 \pm 0.0\%$}
\newcommand{\daSecurityScoreNominal}{2.93}
\newcommand{\daSecurityScore}{$2.93 \pm 0.0$}
\newcommand{\daCurrentMeanInsecurityPercNominal}{68.8\%}
\newcommand{\daCurrentMeanInsecurityPerc}{$68.8 \pm 0.0\%$}
\newcommand{\daCurrentMeanInsecurityPercTwosfNominal}{69\%}
\newcommand{\daCurrentMeanInsecurityPercTwosf}{$69 \pm 0\%$}
\newcommand{\daCurrentMeanSecurityPercNominal}{31.2\%}
\newcommand{\daCurrentMeanSecurityPerc}{$31.2 \pm 0.2\%$}
\newcommand{\daCurrentMeanSecurityPercTwosfNominal}{31\%}
\newcommand{\daCurrentMeanSecurityPercTwosf}{$31 \pm 0\%$}
\newcommand{\daCurrentVulnFreeNominal}{0.312}
\newcommand{\daCurrentVulnFree}{$0.312 \pm 0.002$}
\newcommand{\daCurrentMeanOutstandingVulnerabilities}{$0.0645 \pm 0.0$}
\newcommand{\daCurrentUpdatednessNominal}{0.0934}
\newcommand{\daCurrentUpdatedness}{$0.0934 \pm 0.0019$}
\newcommand{\daCurrentUpdatednessPercNominal}{9.34\%}
\newcommand{\daCurrentUpdatednessPerc}{$9.34 \pm 0.19\%$}
\newcommand{\daCurrentUpdatednessPercTwosfNominal}{9.3\%}
\newcommand{\daCurrentUpdatednessPercTwosf}{$9.3 \pm 0.0\%$}
\newcommand{\daCurrentSecurityScoreNominal}{4.43}
\newcommand{\daCurrentSecurityScore}{$4.43 \pm 0.0$}
\newcommand{\daDailyParticipationNominal}{872}
\newcommand{\daDailyParticipation}{$872 \pm 271$}
\newcommand{\daNumFullVersions}{1\,300}
\newcommand{\daNumSigFullVersions}{101}
\newcommand{\daNumOSVersions}{50}
\newcommand{\daNumSigOSVersions}{28}
\newcommand{\daNumAPIVersions}{16}
\newcommand{\daNumFullVersionUpdates}{4\,670}
\newcommand{\daNumOSVersionUpdates}{3\,940}
\newcommand{\daUpdatesPerYearNominal}{1.27}
\newcommand{\daUpdatesPerYear}{$1.27 \pm 0.01$}
\newcommand{\daUpdatesPerYearTwosfNominal}{1.3}
\newcommand{\daUpdatesPerYearTwosf}{$1.3 \pm 0.0$}
\newcommand{\daNumFullOnlyVersionUpdates}{724}
\newcommand{\daNumSecurityUpdates}{1\,270}
\newcommand{\daNumPossibleSecurityUpdates}{2\,310}
\newcommand{\daOSCurveFitParamFirst}{83.6}
\newcommand{\daOSCurveFitParamSecond}{\num{0.0028}}
\newcommand{\daOSCurveFitRMSE}{0.114}
\newcommand{\daOSCurvePolyRMSE}{0.114}
\newcommand{\daOSCurveSplineRMSE}{0.114}
\newcommand{\daOSCurveHalfDeployedYearsNominal}{0.908}
\newcommand{\daOSCurveHalfDeployedYears}{$0.908 \pm 0.254$}
\newcommand{\daOSCurveHalfDeployedNominal}{332}
\newcommand{\daOSCurveHalfDeployed}{$332 \pm 92$}
\newcommand{\daOSCurveFullDeployedYearsNominal}{3.16}
\newcommand{\daOSCurveFullDeployedYears}{$3.16 \pm 72$}
\newcommand{\daOSCurveFullDeployedNominal}{1\,150}
\newcommand{\daOSCurveFullDeployed}{$1\,150 \pm 265\,000$}
\newcommand{\daAPICurveFitParamFirst}{97.8}
\newcommand{\daAPICurveFitParamSecond}{\num{0.0028}}
\newcommand{\daAPICurveFitRMSE}{0.11}
\newcommand{\daAPICurvePolyRMSE}{0.111}
\newcommand{\daAPICurveSplineRMSE}{0.111}
\newcommand{\daAPICurveHalfDeployedYearsNominal}{0.946}
\newcommand{\daAPICurveHalfDeployedYears}{$0.946 \pm 0.244$}
\newcommand{\daAPICurveHalfDeployedNominal}{345}
\newcommand{\daAPICurveHalfDeployed}{$345 \pm 89$}
\newcommand{\daAPICurveFullDeployedYearsNominal}{3.2}
\newcommand{\daAPICurveFullDeployedYears}{$3.2 \pm 7$}
\newcommand{\daAPICurveFullDeployedNominal}{1\,170}
\newcommand{\daAPICurveFullDeployed}{$1\,170 \pm 265\,000$}
\newcommand{\daVulnAPKDuplicateFileOctoberPerc}{91.7\%}
\newcommand{\daVulnZergRushMonthsDefFixDeployed}{27.3}
\newcommand{\daTabSecScoresmanufacturer}{\begin{table*} \centering \begin{tabular}{l|c|c|c|c} Name & $f$ & $u$ & $m$ & \textbf{score} \\ &&&& (out of 10) \\ \hline LG & $0.24 \pm 0.00$ & $0.34 \pm 0.00$ & $0.60 \pm 0.01$ & $4.11 \pm 0.02$ \\  Motorola & $0.19 \pm 0.00$ & $0.12 \pm 0.00$ & $0.71 \pm 0.02$ & $3.11 \pm 0.02$ \\  Samsung & $0.13 \pm 0.00$ & $0.04 \pm 0.00$ & $0.58 \pm 0.00$ & $2.80 \pm 0.00$ \\  Sony & $0.14 \pm 0.00$ & $0.19 \pm 0.00$ & $1.09 \pm 0.02$ & $2.66 \pm 0.02$ \\  HTC & $0.14 \pm 0.00$ & $0.10 \pm 0.00$ & $0.86 \pm 0.01$ & $2.65 \pm 0.02$ \\  asus & $0.21 \pm 0.00$ & $0.51 \pm 0.01$ & $6.01 \pm 0.07$ & $2.38 \pm 0.02$ \\  other & $0.06 \pm 0.00$ & $0.05 \pm 0.00$ & $1.00 \pm 0.01$ & $2.02 \pm 0.02$ \\  alps & $0.03 \pm 0.00$ & $0.19 \pm 0.01$ & $3.99 \pm 0.08$ & $0.80 \pm 0.02$ \\  Symphony & $0.00 \pm 0.00$ & $0.09 \pm 0.00$ & $5.00 \pm 0.05$ & $0.32 \pm 0.01$ \\  walton & $0.00 \pm 0.00$ & $0.09 \pm 0.00$ & $6.00 \pm 0.08$ & $0.27 \pm 0.01$ \\ \end{tabular} \caption{Security scores for manufacturers} \label{tab:sec_manufacturer} \end{table*}}
\newcommand{\daSecScoreBestmanufacturer}{LG}
\newcommand{\daSecScoreBestmanufacturerScoreNominal}{4.11}
\newcommand{\daSecScoreBestmanufacturerScore}{$4.11 \pm 0.0$}
\newcommand{\daSecScoreBestmanufacturerNumFullVersions}{142}
\newcommand{\daSecScoreWorstmanufacturer}{walton}
\newcommand{\daSecScoreWorstmanufacturerScoreNominal}{0.273}
\newcommand{\daSecScoreWorstmanufacturerScore}{$0.273 \pm 0.007$}
\newcommand{\daSecScoreWorstmanufacturerNumFullVersions}{21}
\newcommand{\daTabSecScoresmodel}{\begin{table*} \centering \begin{tabular}{l|c|c|c|c} Name & $f$ & $u$ & $m$ & \textbf{score} \\ &&&& (out of 10) \\ \hline Galaxy Nexus & $0.50 \pm 0.00$ & $0.54 \pm 0.01$ & $1.53 \pm 0.04$ & $4.70 \pm 0.04$ \\  Nexus 4 & $0.33 \pm 0.00$ & $0.82 \pm 0.01$ & $6.06 \pm 0.09$ & $3.78 \pm 0.04$ \\  Nexus 7 & $0.27 \pm 0.00$ & $0.73 \pm 0.01$ & $5.92 \pm 0.09$ & $3.29 \pm 0.04$ \\  other & $0.10 \pm 0.00$ & $0.14 \pm 0.00$ & $0.51 \pm 0.00$ & $3.08 \pm 0.00$ \\  Desire HD & $0.08 \pm 0.00$ & $0.05 \pm 0.00$ & $0.38 \pm 0.02$ & $2.91 \pm 0.04$ \\  HTC Sensation & $0.35 \pm 0.00$ & $0.01 \pm 0.01$ & $1.57 \pm 0.05$ & $2.44 \pm 0.05$ \\  GT-I9100 & $0.22 \pm 0.00$ & $0.02 \pm 0.00$ & $1.23 \pm 0.02$ & $2.27 \pm 0.02$ \\  HTC Desire S & $0.02 \pm 0.00$ & $0.02 \pm 0.00$ & $1.00 \pm 0.06$ & $1.74 \pm 0.07$ \\  GT-N7000 & $0.25 \pm 0.00$ & $0.00 \pm 0.00$ & $2.52 \pm 0.05$ & $1.43 \pm 0.02$ \\  GT-P1000 & $0.01 \pm 0.00$ & $0.00 \pm 0.01$ & $1.79 \pm 0.06$ & $0.90 \pm 0.05$ \\  GT-I9505 & $0.06 \pm 0.00$ & $0.13 \pm 0.00$ & $6.82 \pm 0.07$ & $0.62 \pm 0.01$ \\  GT-I9300 & $0.13 \pm 0.00$ & $0.01 \pm 0.00$ & $6.23 \pm 0.04$ & $0.57 \pm 0.01$ \\  HTC Desire HD & $0.00 \pm 0.00$ & $0.00 \pm 0.01$ & $3.03 \pm 0.05$ & $0.28 \pm 0.03$ \\  GT-N7100 & $0.06 \pm 0.00$ & $0.00 \pm 0.01$ & $6.93 \pm 0.08$ & $0.24 \pm 0.02$ \\  Symphony W68 & $0.00 \pm 0.00$ & $0.00 \pm 0.01$ & $11.00 \pm 0.12$ & $0.00 \pm 0.03$ \\ \end{tabular} \caption{Security scores for models} \label{tab:sec_model} \end{table*}}
\newcommand{\daSecScoreBestmodel}{Galaxy Nexus}
\newcommand{\daSecScoreBestmodelScoreNominal}{4.7}
\newcommand{\daSecScoreBestmodelScore}{$4.7 \pm 0.0$}
\newcommand{\daSecScoreBestmodelNumFullVersions}{49}
\newcommand{\daSecScoreWorstmodel}{Symphony W68}
\newcommand{\daSecScoreWorstmodelScoreNominal}{0.0001}
\newcommand{\daSecScoreWorstmodelScore}{$0.0001 \pm 0.0272$}
\newcommand{\daSecScoreWorstmodelNumFullVersions}{1}
\newcommand{\daTabSecScoressummary}{\begin{table*} \centering \begin{tabular}{l|c|c|c|c} Name & $f$ & $u$ & $m$ & \textbf{score} \\ &&&& (out of 10) \\ \hline nexus & $0.40 \pm 0.00$ & $0.49 \pm 0.00$ & $0.55 \pm 0.01$ & $5.28 \pm 0.02$ \\  non-nexus & $0.11 \pm 0.00$ & $0.03 \pm 0.00$ & $0.51 \pm 0.00$ & $2.76 \pm 0.00$ \\ \end{tabular} \caption{Security scores for nexus} \label{tab:sec_summary} \end{table*}}
\newcommand{\daTabSecScoresoperator}{\begin{table*} \centering \begin{tabular}{l|c|c|c|c} Name & $f$ & $u$ & $m$ & \textbf{score} \\ &&&& (out of 10) \\ \hline O2 uk & $0.28 \pm 0.00$ & $0.12 \pm 0.00$ & $0.37 \pm 0.02$ & $3.91 \pm 0.03$ \\  T-Mobile & $0.22 \pm 0.00$ & $0.19 \pm 0.00$ & $0.39 \pm 0.01$ & $3.87 \pm 0.02$ \\  Orange & $0.22 \pm 0.00$ & $0.10 \pm 0.00$ & $0.36 \pm 0.02$ & $3.64 \pm 0.04$ \\  3 & $0.22 \pm 0.00$ & $0.09 \pm 0.00$ & $0.47 \pm 0.02$ & $3.45 \pm 0.03$ \\  Sprint & $0.18 \pm 0.00$ & $0.11 \pm 0.00$ & $0.43 \pm 0.02$ & $3.41 \pm 0.03$ \\  Vodafone uk & $0.14 \pm 0.00$ & $0.13 \pm 0.00$ & $0.52 \pm 0.03$ & $3.19 \pm 0.04$ \\  AT\&T & $0.15 \pm 0.00$ & $0.08 \pm 0.00$ & $0.43 \pm 0.02$ & $3.17 \pm 0.02$ \\  unknown & $0.12 \pm 0.00$ & $0.20 \pm 0.00$ & $0.80 \pm 0.01$ & $2.95 \pm 0.02$ \\  Verizon & $0.19 \pm 0.00$ & $0.09 \pm 0.00$ & $0.82 \pm 0.02$ & $2.86 \pm 0.02$ \\  n Telenor & $0.05 \pm 0.00$ & $0.12 \pm 0.00$ & $1.14 \pm 0.02$ & $2.01 \pm 0.02$ \\  Airtel & $0.05 \pm 0.00$ & $0.03 \pm 0.00$ & $1.44 \pm 0.03$ & $1.45 \pm 0.03$ \\  Grameenphone & $0.01 \pm 0.00$ & $0.04 \pm 0.00$ & $1.66 \pm 0.02$ & $1.10 \pm 0.01$ \\  Robi & $0.00 \pm 0.00$ & $0.08 \pm 0.00$ & $2.08 \pm 0.04$ & $0.91 \pm 0.03$ \\  banglalink & $0.00 \pm 0.00$ & $0.04 \pm 0.00$ & $2.55 \pm 0.04$ & $0.55 \pm 0.02$ \\ \end{tabular} \caption{Security scores for operators} \label{tab:sec_operator} \end{table*}}
\newcommand{\daSecScoreBestoperator}{O2 uk}
\newcommand{\daSecScoreBestoperatorScoreNominal}{3.91}
\newcommand{\daSecScoreBestoperatorScore}{$3.91 \pm 0.0$}
\newcommand{\daSecScoreBestoperatorNumFullVersions}{103}
\newcommand{\daSecScoreWorstoperator}{banglalink}
\newcommand{\daSecScoreWorstoperatorScoreNominal}{0.547}
\newcommand{\daSecScoreWorstoperatorScore}{$0.547 \pm 0.017$}
\newcommand{\daSecScoreWorstoperatorNumFullVersions}{75}
\newcommand{\daTabDLDistances}{\begin{table} \centering \begin{tabular}{l|c|c|c|c} & manufacturer & model & operator & nexus \\ \hline equal & 0.0 & 0.0 & 0.0714 & 0.0 \\weight $m$ & 0.2 & 0.333 & 0.214 & 0.0 \\weight $u$ & 0.2 & 0.0 & 0.143 & 0.0 \\$f$ & 0.3 & 0.733 & 0.357 & 0.0 \\$m$ & 0.6 & 0.733 & 0.357 & 0.5 \\$u$ & 0.9 & 0.667 & 0.786 & 0.0 \\\end{tabular} \caption{Normalised Damerau-Levenshtein distances for different metrics} \label{tab:dl_distances} \end{table}}
\newcommand{\daTabChangeInScores}{\begin{table*}\small \centering \begin{tabular}{l|c|c|c|c} & manufacturer & model & operator & nexus \\ \hline $m$ & $-1.67 \pm 2.26$ & $-0.65 \pm 2.56$ & $-3.34 \pm 1.27$ & $-3.4 \pm 1.9$ \\weight $m$ & $-0.263 \pm 0.278$ & $-0.0958 \pm 0.325$ & $-0.464 \pm 0.159$ & $-0.488 \pm 0.225$ \\equal & $-0.106 \pm 0.061$ & $-0.0343 \pm 0.111$ & $-0.145 \pm 0.043$ & $-0.164 \pm 0.035$ \\weight $u$ & $-0.0569 \pm 0.121$ & $-0.00706 \pm 0.225$ & $0.029 \pm 0.082$ & $-0.00391 \pm 0.12$ \\$u$ & $0.389 \pm 1.62$ & $0.238 \pm 2.09$ & $1.6 \pm 0.9$ & $1.44 \pm 1.51$ \\$f$ & $0.958 \pm 0.556$ & $0.309 \pm 1.0$ & $1.31 \pm 0.39$ & $1.48 \pm 0.32$ \\\end{tabular} \caption{Mean change in scores for different metrics} \label{tab:change_in_scores} \end{table*}}
\newcommand{\daGPAPICurveFitParamFirst}{82.1}
\newcommand{\daGPAPICurveFitParamSecond}{\num{0.00257}}
\newcommand{\daGPAPICurveFitRMSE}{0.176}
\newcommand{\daGPAPICurvePolyRMSE}{0.176}
\newcommand{\daGPAPICurveSplineRMSE}{0.176}
\newcommand{\daGPAPICurveHalfDeployedYearsNominal}{0.964}
\newcommand{\daGPAPICurveHalfDeployedYears}{$0.964 \pm 0.462$}
\newcommand{\daGPAPICurveHalfDeployedNominal}{352}
\newcommand{\daGPAPICurveHalfDeployed}{$352 \pm 169$}
\newcommand{\daGPAPICurveFullDeployedYearsNominal}{3.42}
\newcommand{\daGPAPICurveFullDeployedYears}{$3.42 \pm 79$}
\newcommand{\daGPAPICurveFullDeployedNominal}{1\,250}
\newcommand{\daGPAPICurveFullDeployed}{$1\,250 \pm 289\,000$}
\newcommand{\daGPAPIPerAPIMeanNominal}{0.00472}
\newcommand{\daGPAPIPerAPIMean}{$0.00472 \pm 0.00327$}
\newcommand{\daGPAPIPerAPIStdevNominal}{0.0209}
\newcommand{\daGPAPIPerAPIStdev}{$0.0209 \pm 0.0121$}
\newcommand{\daGPAPIPerAPIAbsMeanNominal}{0.0143}
\newcommand{\daGPAPIPerAPIAbsMean}{$0.0143 \pm 0.0066$}
\newcommand{\daGPAPIPerAPIAbsStdevNominal}{0.0155}
\newcommand{\daGPAPIPerAPIAbsStdev}{$0.0155 \pm 0.0112$}
\newcommand{\daGPAPIPerAPIParametersTable}{\begin{table} \centering \begin{tabular}{c|S|S} API Version & ${t_0}$ & $\mathrm{decay}$ \\ \hline2	&	0.0	&	0.0237\\3	&	0.0	&	0.0235\\4	&	0.0	&	0.00689\\5	&	36.3	&	0.00429\\6	&	0.0	&	0.0043\\7	&	0.0	&	0.00499\\8	&	47.7	&	0.00415\\9	&	155	&	0.00365\\10	&	90.3	&	0.00364\\11	&	445	&	0.00236\\12	&	356	&	0.00229\\13	&	305	&	0.00237\\14	&	219	&	0.0024\\15	&	161	&	0.00241\\16	&	143	&	0.00229\\17	&	292	&	0.00199\\18	&	136	&	0.00152\\19	&	123	&	0.00158\\\end{tabular} \caption{Parameters for different API versions} \label{tab:per_api_parameters} \end{table}}
\newcommand{\daGPAPISeventeenLaterDate}{2015-09-07}
\newcommand{\daGPAPISeventeenLaterProportion}{79.9\%}
\newcommand{\daGPAPISeventeenEarlierProportion}{20.1\%}
\newcommand{\daAPISeventeenReleaseDate}{2012-10-29}
\newcommand{\daGPAPISeventeenFullDeployment}{2017-09-28}
\newcommand{\daGPAPISeventeenCurveFitParamFirst}{292}
\newcommand{\daGPAPISeventeenCurveFitParamSecond}{\num{0.00199}}
\newcommand{\daGPAPISeventeenCurveFitRMSE}{0.027}
\newcommand{\daGPAPISeventeenCurvePolyRMSE}{0.0103}
\newcommand{\daGPAPISeventeenCurveSplineRMSE}{0.0103}
\newcommand{\daGPAPISeventeenCurveHalfDeployedYearsNominal}{1.75}
\newcommand{\daGPAPISeventeenCurveHalfDeployedYears}{$1.75 \pm 0.07$}
\newcommand{\daGPAPISeventeenCurveHalfDeployedNominal}{639}
\newcommand{\daGPAPISeventeenCurveHalfDeployed}{$639 \pm 27$}
\newcommand{\daGPAPISeventeenCurveFullDeployedYearsNominal}{4.92}
\newcommand{\daGPAPISeventeenCurveFullDeployedYears}{$4.92 \pm 1.07$}
\newcommand{\daGPAPISeventeenCurveFullDeployedNominal}{1\,800}
\newcommand{\daGPAPISeventeenCurveFullDeployed}{$1\,800 \pm 390$}
\newcommand{\daNumUpdatesUpgrades}{3\,690}
\newcommand{\daUpdatesPerMonthPerVersionNominal}{2.59}
\newcommand{\daUpdatesPerMonthPerVersion}{$2.59 \pm 0.04$}
\newcommand{\daNumUpdatesSkippedBig}{3}
\newcommand{\daNumUpdatesBigUpgrades}{843}
\newcommand{\daNumUpdatesDowngrades}{146}
\newcommand{\daNumUpdateFullOnly}{721}
\newcommand{\daPercBigUpgradesNominal}{18.5\%}
\newcommand{\daPercBigUpgrades}{$18.5 \pm 0.6\%$}
\newcommand{\daPercUpdatesDowngradesNominal}{3.2\%}
\newcommand{\daPercUpdatesDowngrades}{$3.2 \pm 0.2\%$}
\newcommand{\daNumDataPoints}{103 billion}
\newcommand{\daMonthsDevices}{2\,140}
\newcommand{\daMonths}{6}
\newcommand{\daAdbEnabledPercNominal}{19.2\%}
\newcommand{\daAdbEnabledPerc}{$19.2 \pm 0.0\%$}
\newcommand{\daNumStartedApps}{129\,000}
\newcommand{\daNumInstalledApps}{184\,000}
\newcommand{\daAPISeventeenReleaseDateMonth}{October 2012}
\newcommand{\daGPAPISeventeenFullDeploymentMonth}{September 2017}
\newcommand{\daMeanOutstandingVulnerabilitiesNominal}{0.509}
\newcommand{\daCurrentMeanOutstandingVulnerabilitiesNominal}{0.0645}
\newcommand{\daModelHalfDeploymentYears}{0.0822}
\newcommand{\daModelHalfDeployment}{30}
\newcommand{\daModelFullDeploymentYears}{0.888}
\newcommand{\daModelFullDeployment}{324}
\newcommand{\daMeanOutstandingProbability}{1.0\%}
\newcommand{\daMeanOutstandingUncertaintyEquation}{$0.99^n$}
\newcommand{\daGPAPISeventeenLaterDateMonth}{September 2015}
\newcommand{\daTabSpearmanRanks}{\begin{table} \centering \begin{tabular}{l|c|c|c|c} & manufacturer & model & operator & nexus \\ $\pm\sigma$ & 0.211 & 0.169 & 0.175 & 0.632 \\ \hline $u$ & 0.309 & 0.8 & 0.609 & 1.0 \\$m$ & 0.794 & 0.589 & 0.965 & -1.0 \\$f$ & 0.842 & 0.754 & 0.943 & 1.0 \\weight $m$ & 0.976 & 0.968 & 0.987 & 1.0 \\weight $u$ & 0.976 & 1.0 & 0.991 & 1.0 \\equal & 1.0 & 1.0 & 0.996 & 1.0 \\\end{tabular} \caption{Spearman Rank correlation coefficients for different metrics. The uncertainty is constant for each column but does not take into account the uncertainty in the score which produced the ranking.} \label{tab:spearman_ranks} \end{table}}
\newcommand{\daStartDateMonth}{July 2011}
\newcommand{\daEndDateMonth}{September 2015}
\newcommand{\daSecScoreBestsummary}{nexus}
\newcommand{\daSecScoreBestsummaryScoreNominal}{5.28}
\newcommand{\daSecScoreBestsummaryScore}{$5.28 \pm 0.0$}
\newcommand{\daSecScoreBestsummaryNumFullVersions}{99}
\newcommand{\daSecScoreWorstsummary}{non-nexus}
\newcommand{\daSecScoreWorstsummaryScoreNominal}{2.76}
\newcommand{\daSecScoreWorstsummaryScore}{$2.76 \pm 0.0$}
\newcommand{\daSecScoreWorstsummaryNumFullVersions}{1\,280}

\newcommand{\da}{Device Analyzer}
\newcommand{\dafoot}{\textsuperscript{\ref{foot:dadata}}}
\newcommand{\avoNumSubmitters}{7}
\newcommand{\avoTotalExternalLines}{25 Million}
\newcommand{\avoNumExternalProjects}{176}
\newcommand{\avoNumBigExternalLinesOfCode}{25 Million}
\newcommand{\avoBigExternalLinesOfCodePerc}{99.7\%}
\newcommand{\avoNumVulnerabilities}{31}
\newcommand{\avoNumVulnAllAndroid}{15}
\newcommand{\avoNumVulnSpecific}{16}
\newcommand{\avoStartDate}{2013-08-28}
\newcommand{\avoEndDate}{2014-10-17}
\newcommand{\avoFirstDataDate}{2010-07-13}
\newcommand{\avoLastDataDate}{2014-07-29}
\newcommand{\avoVulnsPerYearNominal}{7.66}
\newcommand{\avoVulnsPerYear}{$7.66 \pm 1.38$}
\newcommand{\avoVulnsPerYearAllAndroidNominal}{3.71}
\newcommand{\avoVulnsPerYearAllAndroid}{$3.71 \pm 0.95$}
\newcommand{\avoTabAndVulns}{\begin{table} \centering \small \begin{tabular}{l|l|c|c} Vulnerability & How known & Date & Categories\\ \hline KillingInTheNameOf & Fixed on & 2010-07-13 & system, kernel\\ exploid udev & Discovered on & 2010-07-15 & kernel\\ levitator & Discovered on & 2011-03-10 & kernel\\ Gingerbreak & Fixed on & 2011-04-18 & system\\ zergRush & Discovered on & 2011-10-06 & system\\ APK duplicate file & Discovered on & 2013-02-18 & signature\\ APK unchecked name & Discovered on & 2013-06-30 & signature\\ APK unsigned shorts & Fixed on & 2013-07-03 & signature\\ vold asec & Fixed on & 2014-01-27 & system\\ Fake ID & Fixed on & 2014-04-17 & signature\\ TowelRoot & Discovered on & 2014-05-03 & kernel\\\end{tabular} \caption{Critical vulnerabilities in Android} \label{tab:andvulns} \end{table}}
\newcommand{\avoNumBigExternalProjects}{40}
\newcommand{\avoNumAnalysedExternalProjects}{28}
\newcommand{\avoNumAnalysedExternalLinesOfCode}{6 Million}
\newcommand{\avoAnalysedExternalLinesOfCodePerc}{24.9\%}
\newcommand{\avoBigExternalMedianVersions}{1.5}
\newcommand{\avoBigExternalMeanVersionsNominal}{2.07}
\newcommand{\avoBigExternalMeanVersions}{$2.07 \pm 1.44$}
\newcommand{\avoBigExternalTotalVersions}{58}

\newcommand{\avo}{AVO}
\newcommand{\percMarketShare}{83.6\%~\footnote{\url{http://www.theinquirer.net/inquirer/news/2379036/android-hits-836-percent-marketshare-while-ios-windows-and-blackberry-slide}}}
\newcommand{\daNumDevices}{\daNumOSDevices}
\newcommand{\daDeviceDays}{\daOSTotalDaysData}
% Num versions since \daStartDate
\newcommand{\opensslNumVersions}{51}
\newcommand{\linuxNumVersions}{478}
\newcommand{\linuxMeanUpdateLatency}{$137 \pm 48.9$}
\newcommand{\opensslMeanUpdateLatency}{$108 \pm 63.6$}
\newcommand{\bouncycastleNumVersions}{5}
\newcommand{\bouncycastleMeanUpdateLatency}{$220 \pm 70.4$}
\newcommand{\linuxMeanUpdateLatencyNominal}{137}
\newcommand{\opensslMeanUpdateLatencyNominal}{108}
\newcommand{\bouncycastleMeanUpdateLatencyNominal}{220}

\newcommand{\otherProjNum}{\avoNumExternalProjects}%TODO check we are doing the right calculation here and not overcounting

% Blinding function
\newcommand{\identifying}[1]{#1}%{}%
\newcommand{\blindauthors}[1]{#1}%{Paper \#XX}%

% Evil hackery to provide an optional argument: https://tex.stackexchange.com/questions/308/different-command-definitions-with-and-without-optional-argument/314#314
\makeatletter
 \def\avovuln{\@ifnextchar[{\@avovulnsspecific}{\@avovulngeneral}}
 \def\@avovulnsspecific[#1]#2{\emph{\href{http://androidvulnerabilities.org/vulnerabilities/#1}{#2}}}
 \def\@avovulngeneral#1{\emph{\href{http://androidvulnerabilities.org/vulnerabilities/#1}{#1}}}
\makeatother

% Terminology

% Android, Unix, adb
% update, release, patch: what do we actually mean and what relevance does that have to security?
% Device manufacturers
% Network operators
% Device model TODO check
% Device handset TODO check


\begin{document}
%
% --- Author Metadata here ---

\newfont{\mycrnotice}{ptmr8t at 7pt}
\newfont{\myconfname}{ptmri8t at 7pt}
\let\crnotice\mycrnotice%
\let\confname\myconfname%

\permission{Permission to make digital or hard copies of all or part of this work for personal or classroom use is granted without fee provided that copies are not made or distributed for profit or commercial advantage and that copies bear this notice and the full citation on the first page. Copyrights for components of this work owned by others than the author(s) must be honored. Abstracting with credit is permitted. To copy otherwise, or republish, to post on servers or to redistribute to lists, requires prior specific permission and/or a fee. Request permissions from \href{mailto:Permissions@acm.org}{Permissions@acm.org}.}
\conferenceinfo{SPSM'15,}{October 12, 2015, Denver, Colorado, USA. \\
{\mycrnotice{Copyright is held by the owner/author(s). Publication rights licensed to ACM.}}}
\copyrightetc{ACM \the\acmcopyr}
\crdata{978-1-4503-3819-6/15/10\ ...\$15.00.\\
DOI: \href{http://dx.doi.org/10.1145/2808117.2808118}{http://dx.doi.org/10.1145/2808117.2808118}}
% Update the XXX's to the DOI assigned by ACM rightsreview forms

\clubpenalty=10000
\widowpenalty=10000
% --- End of Author Metadata ---

\title{Security Metrics for the Android Ecosystem}

\numberofauthors{3}
\author{
\href{http://orcid.org/0000-0001-8936-0683}{Daniel~R.~Thomas}
\and
\href{http://orcid.org/0000-0003-0818-6535}{Alastair~R.~Beresford}\\
       \affaddr{Computer Laboratory}\\
       \affaddr{University of Cambridge}\\
       \affaddr{Cambridge, United Kingdom}\\
       \email{Firstname.Lastname@cl.cam.ac.uk}
\and
\href{http://orcid.org/0000-0002-4677-8032}{Andrew~Rice}
}


\maketitle

\begin{abstract}
The security of Android depends on the timely delivery of updates to fix critical vulnerabilities.
In this paper we map the complex network of players in the Android ecosystem who must collaborate to provide updates, and determine that inaction by some manufacturers and network operators means many handsets are vulnerable to critical vulnerabilities.
We define the FUM security metric to rank the performance of device manufacturers and network operators, based on their provision of updates and exposure to critical vulnerabilities.
Using a corpus of \daNumOSDevices\ devices we show that there is significant variability in the timely delivery of security updates across different device manufacturers and network operators.
This provides a comparison point for purchasers and regulators to determine which device manufacturers and network operators provide security updates and which do not.
We find that on average \daMeanInsecurityPercNominal\ of Android devices are exposed to at least one of \daNumVulnsUsed\ known critical vulnerabilities and, across the ecosystem as a whole, assign a FUM security score of \daSecurityScoreNominal\ out of 10. In our data, Nexus devices do considerably better than average with a score of \daSecScoreBestsummaryScoreNominal; and \daSecScoreBestmanufacturer\ is the best manufacturer with a score of \daSecScoreBestmanufacturerScoreNominal.
%On average \daMeanOutstandingVulnerabilitiesNominal\ outstanding vulnerabilities are not fixed on any device and on average only \daUpdatednessPercNominal\ of devices run the current version of Android.
\end{abstract}

% A category with the (minimum) three required fields
\category{Security and privacy}{Systems security}{Operating systems security}[Mobile platform security]
\category{Security and privacy}{Systems security}{Vulnerability management}

\terms{Security, Measurement, Economics}

\keywords{Android; updates; vulnerabilities; metrics; ecosystems}
\vfill\eject
\section{Introduction}

All large software systems today contain undiscovered security vulnerabilities.
Once discovered, these flaws are often exploited, and therefore the timely delivery of security updates is important to protect such systems, particularly when devices are connected to the Internet and therefore can be exploited remotely.
Manufactures and software companies have known about this issue for many years and are expected to provide regular updates to protect their users.
For example, Windows XP could be purchased for a one-off payment in October 2001 and received monthly security updates until support ended in April 2014.

Unfortunately something has gone wrong with the provision of security updates in the Android market. 
Many smartphones are sold on 12--24 month contracts, and yet our data shows few Android devices receive many security updates, with an overall average of just \daUpdatesPerYearNominal\ updates per year, leaving devices unpatched for long periods of time.

In order to improve our understanding, we need to know more about the Android ecosystem as a whole.
It is a complex system with many parties involved in a long multi-stage pipeline~\cite{HTC2013}.
We map and quantify the major players in this space who must collaborate to provide updates~(\S\ref{sec:android_ecosystem}) and determine that inaction~(\S\ref{sec:bottleneck}) by some of the manufacturers and network operators means many handsets are vulnerable to critical vulnerabilities.
Understanding this ecosystem is all the more important because device manufacturers have introduced additional vulnerabilities in the past~\cite{Grace2012}.

Corporate and public sector buyers are encouraged to purchase secure devices, but we have found little concrete guidance on the specific makes and models providing timely security updates.
For example, CESG, which advises the UK government on how to secure its computer systems, recommends picking Android device models from device manufactures that are good at promptly shipping security updates, but it does not state which device manufacturers these are~\cite{CESG2013} and so far they have only certified one Android device model~\cite{CESG2015}.
Similarly, we are collaborating with a FTSE 100 company who wish to know which devices are secure and which manufacturers provide updates.

The difficulty is that the market for Android security today is like the market for lemons: there is information asymmetry between the manufacturer, who knows whether the device is currently secure and will receive security updates, and the customer, who does not.
To address the asymmetry, we develop a scoring system and provide numbers on the historic performance of device models found in the \da~\cite{Wagner2013} project~(\S\ref{sec:security_scoring}).
We propose three metrics:
$f$ the proportion of running devices free from critical vulnerabilities over time;
$u$ the proportion of devices that run the latest version of Android shipped to any device produced by that device manufacturer; and
$m$ the mean number of outstanding vulnerabilities affecting devices not fixed on any device shipped by the device manufacturer.
We then derive a composite FUM score which is hard to game (\S\ref{sec:gaming}).
%We provide the answer to the question ``Which is the most secure Android provider I can buy from?'', showing that in general Nexus devices are better; further details are given in~\S\ref{sec:security_scoring:results}.

The FUM score enables corporate and public sector buyers, as well as individuals, to make more informed purchasing decisions by reducing the information asymmetry.
The FUM score also supports better regulation, and indeed there is ongoing legal action to force network operators to ship updates for security vulnerabilities~\cite{Soghoian2013}.\todolater{Check on the status of this legal action}
We will continue to provide updated versions of our FUM scores on our website~\cite{androidvulnerabilities.org}.

In summary, the contributions of this paper are:
\begin{itemize}
 \item We quantify the Android update process, providing concrete numbers on the flow of updates and their latency~(\S\ref{sec:android_ecosystem}).
 \item We propose the FUM scoring metric to evaluate the security of different instances of a platform~(\S\ref{sec:security_scoring:method}).
 \item We measure the security of Android against our scoring metric and compare different device manufacturers, device models and network operators to allow device purchasers to differentiate between them based on security~(\S\ref{sec:security_scoring:results}).
 \item We determine that the main update bottleneck lies with manufacturers rather than Google, operators or users~(\S\ref{sec:bottleneck}).
\end{itemize}

We indicate the uncertainty in our results by presenting them $\pm$ one standard deviation and give results to 3~s.f., this occasionally results in `$\pm\, 0$' when the standard deviation is small.
\todolater{Do we want to use the 95 percentile instead}
We explore systematic errors in \S\ref{sec:representative}.

\section{Threat model}
\label{sec:threatmodel}

In this paper we are concerned with vulnerabilities which allow an attacker without physical access to the smartphone to gain significant permissions (such as root-level access) which are not available to a standard app running on the device. We consider three attack vectors which can be used as a starting point to launch an attack on a device.

The \emph{installation} attack vector is used when a malicious app is installed on the device.
Android devices can install apps through marketplaces such as the Google Play Store, email attachments, URLs and via the Android Debug Bridge (ADB).
By default, many Android devices will only allow the installation of apps from the Play Store, which automatically analyses apps, and quickly takes down apps that are reported as malicious.
However, alternative markets are also popular, particularly in countries where the Play Store is not available.

The \emph{dynamic code loading} attack vector occurs when an existing app downloads and executes new code at runtime.
The most direct method is to upload a seemingly innocent app to a marketplace that then dynamically loads malicious code, either as additional davik bytecode, as a native library, or by embedding an interpreter and executing received instructions.
Neither static nor dynamic analysis of this app will uncover any malicious code, since it does not exist in the app.
The marketplace can try to detect explicit use of dynamic code loading, however there are ways to dynamically load code which are hard to detect even on a platform such as iOS, which does not permit dynamic code loading.
For example, a Return-Oriented Programming (ROP) attack on iOS is relatively easy if the attacker creates an app with carefully crafted flaws~\cite{Wang2013a}.

The \emph{injection} attack vector occurs when the attacker injects malicious code directly into existing code already running on the handset. 
For example, the addJavascriptInterface vulnerability (CVE-2012-6636) allows an attacker to inject JavaScript into HTTP traffic destined for the device and execute arbitrary code with all the privileges of the app.
The fix for this vulnerability breaks backwards compatibility and requires a two-sided fix.
While the fix was released in December 2012, by \daGPAPISeventeenLaterDateMonth, \daGPAPISeventeenEarlierProportion\ of handsets connecting to the Play Store were still vulnerable to this attack~\cite{Thomas2015a}.

Security for the Android ecosystem can be deployed at three levels: in an online marketplace, at app installation time on the device, and during app execution. 
Google provides its users with security in all these places: through analysis of apps by the Play Store, using the Verify Apps feature on the smartphone at installation time, and by an app sandbox on the smartphone during execution. 
The best place to prevent attacks is by sandboxing the app during execution, since all three attack vectors can be prevented at this level, whereas not all users install apps exclusively via the Play Store or enable Verify Apps.
In addition, dynamic code loading and injection attacks cannot be discovered at installation time and can be difficult for a marketplace to detect.
Unfortunately, as we shall see, the security sandbox for Android has known critical vulnerabilities on most devices.
This does not mean these devices are attacked, but that they are vulnerable.
The likelihood of a successful attack then depends on what apps the user installs and where from, as well as the computer networks the device is connected to and the actions the user takes whilst connected.

\section{Data}
We use two sources of data to measure the security of Android: (1) information on the critical vulnerabilities found to affect particular versions of Android and (2) information on the distribution of Android versions over time.
These two datasets can then be combined to determine the proportion of devices at risk of attack from specific vulnerabilities.

\subsection{Critical vulnerabilities}\label{sec:avo}

We built a list of critical Android vulnerabilities for our AndroidVulnerabilities.org~(\avo) website~\cite{androidvulnerabilities.org}.
The site contains \avoNumVulnerabilities\ critical vulnerabilities such as root vulnerabilities that do not require USB debugging to exploit.
We have chosen \daNumVulnsUsed\ vulnerabilities as shown in Table~\ref{tab:andvulns} for our analysis in this paper.
We selected these vulnerabilities since they fit the attack vectors introduced in \S\ref{sec:threatmodel} and because they affect all Android devices regardless of manufacturer, and as a result our selected vulnerabilities will dominate any security analysis of Android.
Hence, with our chosen set of vulnerabilities, our analysis represents a lower-bound on the vulnerability of devices in the \da\ data.

Some critical vulnerabilities are not traditional kernel vulnerabilities, but exploit the \emph{installation} attack vector in our threat model.
For example improper verification of signatures at installation time was discovered in February 2013~\cite{Forristal2013} and meant that apps could pretend to be signed with system keys and hence be granted system privileges.
On versions of Android below 4.1, malware could then use known system-to-root escalation mechanisms.
Regardless of version, this exposed an increased attack area and would also provide the ability for malware to control all user internet traffic (via VPNs), brick the phone, remove and install apps, steal user credentials and read the screen.
The different categories in which the vulnerabilities fall are shown in Table~\ref{tab:andvulns}. The `signature' vulnerabilities require an \emph{installation} attack, while `kernel' and `system' vulnerabilities can be used together with an \emph{installation}, \emph{dynamic code loading} or \emph{injection} attack vector.
\avoTabAndVulns

%\avo\ currently contains \avoNumVulnerabilities\ vulnerabilities, of which \avoNumVulnAllAndroid\ affect all Android devices and \avoNumVulnSpecific\ are specific to particular devices or device manufacturers.

\subsection{Device Analyzer data}\label{sec:da}
We use historical data collected by the \da\ project~\cite{Wagner2013}.
\da\ collects data\footnote{\url{https://deviceanalyzer.cl.cam.ac.uk/collected.html}} from study participants who install the Android app from the Play Store.
Most study participants allow external researchers to access the subset of the device data needed for this analysis.

We extracted the build string and API version for each device each day.
The build string is a user-readable version string and the API version is a positive integer that increases when new features are added to the API.
Consequently security (bug) fixes do not always result in a change in the API version.
Fortunately most (\daOSVersionPercValidLines) entries in these data have a build string of the form `x.y.z opaque\_marker' and so it is possible to extract the Android version number `x.y.z'.
On a large proportion of devices `opaque\_marker' is a well defined build number\footnote{\url{https://source.android.com/source/build-numbers.html}} however this is not universal.
%Google provides API version distribution information but not the OS and build version information we need.

\da\ has collected data from \daNumDevices\ devices with a total of \daDeviceDays\ device days.
The majority of devices only contribute data for a short period of time, however \daMonthsDevices\ devices have contributed data for more than \daMonths~months.
We verify that the \da\ data is representative in~\S\ref{sec:representative}.

\section{Android Ecosystem}\label{sec:android_ecosystem}
There is a complex Android ecosystem that creates and distributes updates which fix vulnerabilities.
In this section we describe how the Android ecosystem functions and how Android versions are produced, using \da\ data and by analysing the Android source code and upstream projects.
We quantify the number of updates shipped by various entities in the ecosystem and the number of entities.


\label{sec:android_update_process}
\begin{figure}[h]
 \centering
 \def\svgwidth{\columnwidth}
 \import{figures/}{update_ecosystem.pdf_tex}
 \caption{Flow of updates between participants in the Android ecosystem.
 Numbers on edges indicate updates shipped between \daStartDateMonth\ and \daEndDateMonth, those in brackets represent number of such entities.
 Dotted arrows indicate flows where we can't measure because no public data is available.}
 \label{fig:update_ecosystem}
\end{figure}
\begin{table}
\centering
\normalsize
\begin{tabular}{l|r|r}
Project	&	\# releases	&	latency (days) \\ \hline
linux	&	\linuxNumVersions	&	\linuxMeanUpdateLatency \\
openssl	&	\opensslNumVersions	&	\opensslMeanUpdateLatency \\
bouncycastle	&	\bouncycastleNumVersions	&	\bouncycastleMeanUpdateLatency \\
\end{tabular}
\caption{Flow of updates from upstream projects into Android. Number of updates as in Figure~\ref{fig:update_ecosystem}, latency in days between the upstream release and the release of the first Android version containing it, for all pairs of versions we have data on.\todolater{scrape the other 26 websites... is it worth it?}}
\label{tab:update_ecosystem}
\end{table}
To understand how vulnerabilities in Android are fixed we examine the Android update process, which we model in Figure~\ref{fig:update_ecosystem}.
There are five entities or groups that contribute towards Android updates: the network operators, the device manufacturers, the hardware developers, Google and the upstream open source projects.
Android builds on various open source projects, such as the Linux kernel, OpenSSL and BouncyCastle cryptography libraries.
Consequently Android can include any compatible versions of those projects, including those that fix security vulnerabilities.
Android also incorporates various drivers for different bits of hardware.
The Android platform is then built from these components by Google.
The code for each Android release or update is kept secret until after a binary release has been published.\footnote{\url{https://source.android.com/source/code-lines.html}}
Device manufacturers receive advanced access in order to prepare handsets.
The network operator may then make or request customisations and perform testing before shipping the update to the device.
Sometimes device manufactures ship updates directly to the user without involving the network operator.
Sometimes the device manufacturer and Google collaborate closely to make a particular phone, such as with Nexus devices and so Google ships directly to the device.
Sometimes device manufacturers incorporate upstream open source project releases directly, and sometimes incorrectly -- for example previous work has recorded evidence of broken nightly builds of sqlite in Android releases on some device models~\cite{Wagner2013}.

The numbers of devices (\daNumOSDevices), network operators (\daNumOperators) and device manufacturers (\daNumManufacturers) in Figure~\ref{fig:update_ecosystem} come from the \da\ data.
Device manufacturer and network operator counts were obtained by normalising the results reported by Android to \da\ of the device manufacturer and active network operator.
This normalisation is a manual task that involves removing invalid values (such as `manufacturer' or `airplane mode is on'), collating across company name changes (e.g.\ `lge' to `LG'), normalising punctuation, removing extra strings sometimes added such as (`(2g)' or `communications') and mapping some incorrectly placed model names back to their manufacturer.
This normalisation is not perfect so these are likely overestimates on the \da\ data.
We believe they nevertheless are likely to underestimate the total number of device manufacturers and network operators worldwide.
%The representativeness of \da\ is discussed in \S\ref{sec:representative}.

In Figure~\ref{fig:update_ecosystem} the number of updates received by devices (\daNumFullVersions) is the number of different full version strings observed in \da.
The number of updates shipped by Google (\daNumSigOSVersions) is the number of Android versions reported in \da\ that affected more than \daSigVersionPerc\ of devices for more than \daSigVersionDays\ days.
This significance test is to remove spurious versions recorded in \da\ such as `5.2.0' in 2012 which had still not been released at time of writing.\todolater{Check if this has now been released}

We extracted data on the external projects used in Android and have included this and the scripts which generated it in \avo.
These scripts analysed the Android Open Source Project's source tree to examine the source code of each of the external projects to find the project version associated with each Android version tag on the repository.
There are \avoNumExternalProjects\ external open source projects in Android, contributing \avoTotalExternalLines\ lines of code.
We analysed the top \avoNumBigExternalProjects\ by lines of code (\avoBigExternalLinesOfCodePerc\ of the total) and were able to automatically extract the versions of those projects included in different versions of Android for \avoNumAnalysedExternalProjects\ of these (\avoAnalysedExternalLinesOfCodePerc\ of the total).
We found \avoBigExternalTotalVersions\ distinct versions, a median of \avoBigExternalMedianVersions\ and mean of \avoBigExternalMeanVersions\ versions per project.
Android rarely changes the version of external projects it includes.

%An analysis by Vidas et al.~\cite{Vidas2011} of the Android 2.1 to 2.2 update found that it took 11 months from when Google released 2.2 for the last device which they were investigating to get the update.
To compute the latency between upstream releases and the release of the first version of Android containing that release we scraped the release pages, to obtain the version numbers and release dates.
This allows us to compute the latency between an upstream project being released and it being included in Android; this is shown in Table~\ref{tab:update_ecosystem}.
The versions included in Android were about half a year old when the first version of Android containing it was released.


\subsection{Scoring for security}
\label{sec:security_scoring}

Computing how good a particular manufacturer or device model is from a security standpoint is difficult as it depends on a number of factors which are hard to observe, particularly on a large scale.
Ideally we would consider both prevelance of potential problems which were not exploited and actual security failures.
However in the absence of such data we propose a scheme for assigning a device a score out of ten based on data which can be observed, and which hopefully correlates with the actual security of the devices.

This score is computed from several components:
\begin{description}
  \item[$i$] The proportion of the time were devices exposed to known vulnerabilities. This is equivalent to Acer and Jackson's proposal~\cite{Acer2010} to measure the security based on the proportion of users with at least one unpatched critical vulnerability and is also $1 -$ the Vulnerability Free Days score~\cite{Wright2014}.
  \item[$u$] The proportion of devices run the latest version of Android shipped to any device produced by that manufacturer. This is a measure of internal updatedness, a low score would mean many devices are being left behind.
  \item[$m$] The mean number of outstanding vulnerabilities affecting devices not fixed on any device shipped by the manufacturer. This is related to the Median Active Vulnerabilities measure~\cite{Wright2014} but is the mean rather than the median.
%TODO should we compute the median instead?
\end{description}
\begin{equation}
\mathrm{score} = (1 - i)\times 4 + u \times 3 + \frac{2}{1+e^m} \times 3
\end{equation}

The scores accross the whole of android are that \daMeanInsecurityPerc\ of devices are exposed to known root exploits.
There are on average \daMeanOutstandingVulnerabilities\ outstanding vulnerabilities not fixed on any device.
Only on average \daUpdatednessPerc\ of devices run the most recent version of Android.
This gives a security score of \daSecurityScore/10.
\daTabSecScoressummary
However there are a wide variety of scores depending on the source of the device.
There have been many reports that Google's Nexus devices are better at getting updates than other Android devices because Google makes the original updates and ships them to its devices.
Table~\ref{tab:sec_summary} shows that this is the case with Nexus devices getting much better scores than non-Nexus devices.
\daTabSecScoresmanufacturer
Different manufacturers have very different scores, Table~\ref{tab:sec_manufacturer} shows the scores for the \daNumSigManufacturers\ manufacturers with a significant presence in our data.
Manufacturers are considered significant if we have data from at least \daSigNumDevices\ devices and at least \daSigNumDays\ days of contributions.
\daTabSecScoresmodel
Even within manufacturers different models can have very different update behaviours and hence security.
Table~\ref{tab:sec_model} shows the results for the \daNumSigModels\ device models which have a significant presence by the same metric.



\section{Comparison with other data}\label{sec:representative}
We compare the Device Analyzer data with three other sources of data and find similar distributions which bound the Device Analyzer data, indicating that it is likely to be typical.
We have obtained comparable data on 5\,290 devices from a multinational FTSE 100 company's mobile device management database, which includes company and employee owned devices, and from 5\,170\,000 matching User-Agent headers on all HTTP traffic for 30\% of Rwanda for a week.
\todo{citation for Rwanda dataset}
\begin{figure}
\centering
\includegraphics[width=\columnwidth]{figures/dists}
\caption{Comparison between FTSE, User-Agent and the corresponding \da\ data, error bars indicate 95\% confidence intervals.}
\label{fig:dists}
\end{figure}
We used the data from the FTSE 100 company for a week in April 2015 and the User-Agent data was collected in February 2015. %between 2015-01-31 and 2015-02-12.% FTSE 2015-04-27
Figure~\ref{fig:dists} shows the proportion of devices running each Android OS version in the two comparison data sets and the comparable periods from \da.
The general pattern this shows is that in the FTSE data newer versions are more popular than in the \da\ data and that in the Rwanda data old versions are more popular.
Therefore the \da\ data on OS versions is bounded by these two data sets.
\todolater{Can we quantify this?}

Unfortunately, there is no ground truth of OS version information.
However, we have collated the API version information that Google has published most months since 2009.\footnote{\url{http://androidvulnerabilities.org/play/historicplaydashboard}}
While API versions are too coarse grained to use for security update detection they are closely related to OS versions.
If the \da\ data on API versions are similar to the Google data on API versions then the \da\ data on OS versions should be representative.
We compared the data from Google and from \da\ and they are similar.
We analysed the difference between the API version data from \da\ and Google Play, normalising for days since the API version was released.
This shows that the \da\ data systematically overestimates the prevalence of new API versions and underestimates the prevalence of old API versions.
This means that the OS version information from \da\ is likely to be overestimating the prevalence of new OS versions and hence our results are likely to be conservative.\todolater{Provide some numbers which quantify this}

\section{Related work}
We assume updates make security better, however the update process for apps, security fixes and OS upgrades also needs to be secure.
Unfortunately, package management systems designed to provide secure updates have been found to contain vulnerabilities~\cite{Cappos2008} and many software update systems fail to authenticate the connection between the device and the update server or do not authenticate the downloaded binaries~\cite{Bellissimo2006}.
Android does authenticate update binaries and Google Play downloads them over a secure connection~\cite{Viennot2014}.
In this paper we have analysed four critical vulnerabilities in the Android app update mechanism: APK unsigned shorts, APK unchecked name, APK duplicate file and Fake ID.
Other work has demonstrated complex and subtle errors exist in the Android app update process.
For example, the process can be exploited to allow apps to gain privilege through `Pileup' vulnerabilities by registering for new permissions before the update which creates that permission is installed~\cite{Xing2014}.

We used \da\ to record data on the OS versions of Android installed on devices.
Nappa et al.\ performed a similar analysis using WINE to record data on installed versions of Windows client programs and used cleaned NVD data rather than \avo\ for vulnerability data.
They did not assign client applications scores~\cite{Nappa2015}.


We use User-Agent strings from Rwandan internet traffic to examine the version distribution of Android OS versions.
User-Agent strings have been used to investigate the timeliness of web browser updates, with at most 80\% of Firefox users running the most recent version~\cite{Frei2008}.
The same analysis was used to show that Chrome's use of silent updates seems to increase uptake of upgrades~\cite{Duebendorfer2010} with 97\% of users running the latest version within 3 weeks of release.
By way of comparison, Android's update process is manual.
The user is notified an update exists, but further action is required, including downloading the update and rebooting the phone to enable installation.
The phone must have sufficient charge to perform the update and the device itself is rendered inoperable during the update process, two factors that might prevent or delay the update process from taking place.
In our data we are unable to determine why a device is not updated. 
It is possible that many updates arrive at handsets, but are simply not installed.
Anecdotal evidence suggests that it is the lack of updates rather the lack of installation, which is the major problem at present.
Our analysis in \S\ref{sec:bottleneck} supports this view.
Nevertheless it is problematic that an operating system update requires a reboot.
Chrome installs the new version side by side with the old one and switches the next time it is restarted.
The same technique would be more difficult on phones with limited storage space (as many cheap Android phones have barely enough space to install just the update) but is a plausible improvement for more high-end devices.
Google is deploying the same silent update technique through Google Play Services,\footnote{\href{http://lifehacker.com/why-google-play-services-are-now-more-important-than-an-975970197}{http://lifehacker.com/why-google-play-services-are-now-more-important-than-an-975970197}} which automatically installs updates for core Google components of Android, this bypasses the device manufacturer and network operator.

As well as supplying security updates promptly, the impact of vulnerabilities can be reduced through security in depth.
In this regard, iOS provides additional safeguards beyond those used in Android, including a pre-distribution review, mandatory code-signing by the manufacturer, and (with the important exception of ROP-based attacks~\cite{Wang2013a}) the technical prohibition of dynamic code loading by an app.
These features, combined with mandatory access controls, has resulted in a lower level of malware affecting iOS when compared to Android~\cite{Felt2011}.

There are continuing efforts to reduce the impact of critical vulnerabilities, both in Android and elsewhere.
SE\-Android~\cite{Smalley2013}, which is included in Android from version 4.1~\cite{jelly-bean-release}, and fully enforcing from version 5.0~\cite{AndroidSecurity2014} claimed to prevent some root vulnerabilities and to reduce the impact of others.
Capability based enforcement systems such as Capsicum~\cite{Watson2010} substantially reduce the capabilities with which an exploit has to try and gain increased privilege and could be included in Linux\footnote{\url{https://github.com/google/capsicum-linux}} and hence Android.

Rather than fixing critical vulnerabilities, security can be obtained by detecting malicious apps and preventing their installation or execution.
Detection strategies include Risk\-Ranker, which classified 3\,281 out of 118\,318 apps (2.8\%) as risky of which 718 (22\%) were malware and 322 (10\%) were previously unknown malware, an infection rate of 0.6\% across multiple markets~\cite{Grace2012a}.
Droid\-Ranger also analysed apps finding 148 out of 182\,823 apps (0.08\%) to be malicious across multiple markets, of which 29 were previously unknown~\cite{Zhou2012a}.
A common technique used by attackers is to include malicious code in repackaged popular apps. 
An\-Darwin uses this insight to detect similar apps, and found 169 out of 265\,359 of all apps studied (0.06\%) were malicious clones~\cite{Crussell2013}.
Other approaches have made further use of malware dependency graphs~\cite{Lindorfer2014} or tried to extract further semantic information to avoid malware being able to avoid detection~\cite{Zhang2014} but simpler approaches based on extracting strings from the binaries can be more effective~\cite{Arp2014} in terms of false positives and false negatives.
There is a lack of ground truth data, a danger in training algorithms based on APK files that other algorithms have already found which might lead to new malware being missed and DroidChameleon showed that existing AntiVirus apps could be fooled by simple permutations of malicious APKs~\cite{Rastogi2013}.

The percentage of Android devices running the current version (\daUpdatednessPercNominal) is much less than the rate ($>90$\%) for Windows XP SP2 computers contacting the Microsoft update servers~\cite{Gkantsidis2006}.
A simple numerical comparison is unfair because only one major OS version was considered in the Microsoft analysis, and data was only collected from computers that contacted the update server (this was the default).
Later data demonstrates the difficulty of upgrading computers between major OS versions, with 27\% of Windows computers running Windows XP in July 2014,\footnote{\url{https://archive.today/PLGxn}} four months after Windows XP stopped receiving security updates.


\section{Conclusion}
The security of Android depends on the timely delivery of updates to fix critical vulnerabilities.
Unfortunately few devices receive prompt updates, with an overall average of \daUpdatesPerYearNominal\ updates per year, leaving devices unpatched for long periods.
We showed that the bottleneck for the delivery of updates in the Android ecosystem rests with the manufacturers, who fail to provide updates to fix critical vulnerabilities.
This arises in part because the market for Android security today is like the market for lemons: there is information asymmetry between the manufacturer, who knows whether the device is currently secure and will receive updates, and the consumer, who does not.

Consequently there is little incentive for manufacturers to provide updates.
To address this issue we developed the FUM security metric to quantify and rank the performance of device manufacturers and network operators, based on their provision of updates and exposure to critical vulnerabilities.
The metric enables purchasers and regulators to determine which device manufacturers and network operators provide updates and which do not.

Using a corpus of \daNumOSDevices\ devices we demonstrated that there is significant variability in the timely delivery of security updates across different device manufacturers and network operators.
We find that on average \daMeanInsecurityPercNominal\ of Android devices are exposed to at least one of \daNumVulnsUsed\ known critical vulnerabilities and, across the ecosystem as a whole, assign a FUM security score of \daSecurityScoreNominal\ out of 10. In our data, Nexus devices do considerably better than average with a score of \daSecScoreBestsummaryScoreNominal; \daSecScoreBestmanufacturer\ is the best manufacturer with a score of \daSecScoreBestmanufacturerScoreNominal.

\section*{Dataset}
Much of the raw and processed data and source code is available, excluding that which might identify individuals~\cite{Thomas2015e}.
Data from \da\ and \avo\ used in this paper is already available.
The Rwanda and FTSE data cannot be made available.

\section*{Acknowledgements}
\identifying{
This work was supported by a Google focussed research award; and the EPSRC [grant number EP/P505445/1].
Thanks to \href{https://www.cl.cam.ac.uk/~sa497/}{Sherif Akoush} and \href{http://orcid.org/0000-0003-0740-8650}{Ripduman Sohan} for supplying the Rwanda data and to staff at the FTSE 100 company for supplying their MDM data.
Thanks to our anonymous reviewers for their insightful comments; Richard Clayton, the mobile security reading group, Anil Madhavapeddy and Oliver Chick for reading early drafts;
David Robertson for statistical advice;
and Laurent Simon, Thomas Coudray, Adrian Taylor, Justin Case, Giant Pune and Khilan Gudka for reporting vulnerabilities.
}
\sloppy
\printbibliography

\end{document}
