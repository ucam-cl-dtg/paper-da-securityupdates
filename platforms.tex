\documentclass[12pt,a4paper]{article}
\usepackage[utf8]{inputenc}
\usepackage{amsmath}
\usepackage{amsfonts}
\usepackage{amssymb}
\usepackage{graphicx}
\usepackage[left=2cm,right=2cm,top=2cm,bottom=2cm]{geometry}
\author{Daniel Thomas}
\title{Platforms for security update research}
\begin{document}
\maketitle

\section{Introduction}
Early computers allowed running code to do anything, later more restrictive systems were developed which had an operating system which could do anything but which ran programs which were restricted in what they could do.
However mostly this was broken down on a per user basis, code running as a particular user could do anything a user could do.
More recently there have been efforts at greater compartmentalisation and more fine grained permissions for applications.
For example the Android permission system restricts what each app can do in a more precise way than installing applications on Linux does.
This fine grained compartmentalisation only works if there are no known vulnerabilities which break it and allow code to do arbitrary things.
Unfortunately this is not the case.

\section{Android}
\section{iOS}
\section{Firefox}
\section{Chrome}
\section{ChromeOS}
\section{JVM}
\section{Xen}
\section{VMWare}

\end{document}