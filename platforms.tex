\documentclass[12pt,a4paper]{article}
\usepackage[utf8]{inputenc}
\usepackage{amsmath}
\usepackage{amsfonts}
\usepackage{amssymb}
\usepackage{graphicx}
\usepackage[left=2cm,right=2cm,top=2cm,bottom=2cm]{geometry}
\usepackage{hyperref}
\RequirePackage[date=terse,isbn=true,doi=true,url=false,maxbibnames=9,backref=true]{biblatex}
\addbibresource{securityupdates.bib}
\renewcommand{\bibfont}{\small}

\AtEveryBibitem{% Clean up the bibtex rather than editing it
 \clearname{editor} % remove editors
}

\author{Daniel Thomas}
\title{Platforms for security update research}
\begin{document}
\maketitle

\section{Introduction}
Early computers allowed running code to do anything, later more restrictive systems were developed which had an operating system which could do anything but which ran programs which were restricted in what they could do.
However mostly this was broken down on a per user basis, code running as a particular user could do anything a user could do.
More recently there have been efforts at greater compartmentalisation and more fine grained permissions for applications.
For example the Android permission system restricts what each app can do in a more precise way than installing applications on Linux does.
This fine grained compartmentalisation only works if there are no known vulnerabilities which break it and allow code to do arbitrary things.
Unfortunately this is not the case.

\section{Mobile operating systems}


\subsection{Android}
In Android each app runs in a separate user account and is granted certain permissions statically after user review on installation.
Apps can use the rich Android API and embed native code which can make use of the Linux API.
The security boundary is the user but there have been many examples where it is possible for an app to exploit a vulnerability in Android and gain root permissions and do anything.
To analyse the proportion of the time deployed Android devices are exposed to vulnerabilities which break their security boundary we can use the Device Analyzer data to look at running versions and the website\footnote{\url{http://androidvulnerabilities.org}} to find out which vulnerabilities applied to which versions at what time.

Android has many vendors which provides some heterogeneity in vulnerabilities but most of these vendors are not as good as Google at shipping updates promptly (perhaps partly because Google is better placed to do so).

\subsection{iOS}
In iOS every process runs as the same user but mandatory access control is used to restrict what processes can do to permitted things.
iOS has additional features that make breaking the security boundary more difficult: Address Space Layout Randomization (ASLR), Data Execution Prevention (DEP), and Sandboxing, iOS also enforces the mandatory App Review and code signing mechanisms.
However it is still possible to execute arbitrary code on the device~\cite{Wang2013a} and hence exploit flaws to Jailbreak the phone.
To analyse update security on iOS we need detailed information on the version of iOS running on deployed devices \emph{from where?} and a database of discovered vulnerabilities which would need to be created.

The only iOS vendor is Apple who have full control over the OS and its distribution this enables them to make prompt changes but a flaw affecting one device will affect all devices.
Since there is no competition between iOS vendors competition with other mobile OSes is the only incentive for innovation.

\section{Web browsers}
There are two sources of arbitrary code execution in web browsers, extensions which typically come through a market and web pages which can come from anywhere.
Version information for browsers is available.\footnote{\url{http://gs.statcounter.com/#desktop-browser_version-ww-monthly-200807-201401}}

\subsection{Firefox}
Firefox can be extended with extensions which are written in javascript and XML, the APIs allow the equivalent arbitrary code to be run as the Firefox process and so there is no real sandboxing of extensions~\cite{Lerner2013}.
There is manual review to detect and remove bad extensions.
Web pages are allowed to run javascript and with html5 are presented with an increasingly rich API they are not allowed to execute arbitrary code on the computer as the user.
Firefox has a nice list of vulnerabilities with severities.\footnote{\url{https://www.mozilla.org/security/known-vulnerabilities/firefox.html}}

\subsection{Chrome}
Chrome uses privilege separation to separate the renderer from the browser kernel by running them in different processes~\cite{Barth2008}.
Extensions are also split into multiple processes with content scripts running in the host page's renderer process and the extension core running in a separate renderer process~\cite{Liu2012}.
Malicious extensions can inject adverts and steal data or engage in more nefarious attacks on the user but a vulnerability in the sandbox would be required for them to execute arbitrary code.
To analyse the update security of Chrome we need data on the version distribution of Chrome over time and a database of vulnerabilities in Chrome.
There are a fair number of CVEs listed for Chrome.\footnote{\url{http://www.cvedetails.com/vulnerability-list.php?vendor_id=1224&product_id=15031}}
There are some statistics on version distributions for chrome.\footnote{\url{http://www.w3schools.com/browsers/browsers_chrome.asp}}

\subsection{ChromeOS}
ChromeOS takes the same functionality as Chrome and sticks a more secure OS underneath with better sandboxing, process separation, boot verification and secure updates.\footnote{\url{http://www.chromium.org/chromium-os/chromiumos-design-docs/security-overview}}
There might still be security flaws particular to ChromeOS but it seems likely that most flaws would have to be in Chrome first.

\section{Java Virtual Machine (JVM)}
The Java Virtual Machine has a security manager and attempts to impose fine grained permissions on executing Java code~\cite{Gong1997} this is used by Java plugins in browsers to execute remote code without exposing the user to attacks.
Unfortunately there are many flaws in the JVM which have allowed arbitrary execute of code bypassing the security manager.
There are a list of vulnerabilities in the JVM.\footnote{\url{http://www.cvedetails.com/vulnerability-list/vendor_id-5/product_id-1526/opec-1/SUN-JRE.html}}
We also need the distribution of running versions over time.

\section{Hypervisors}
Hypervisors try to improve security by offering a much smaller API to the VM's operating system and enforcing privilege separation by using different CPU ring levels.
Unfortunately they still have vulnerabilities which break this confinement.

\subsection{Xen}
Xen is an opensource hypervisor with a commercial version produced by Citrix.
It is widely used by lots of companies including Amazon which allow multiple mutually untrusting VMs to run on the same machine.
There is are list of vulnerabilities in Xen.\footnote{\url{http://wiki.xenproject.org/wiki/Security_Announcements_(Historical)} \url{http://xenbits.xen.org/xsa/}}
Where do we get version distribution information from?

\subsection{VMWare}
VMWare is a closed source hypervisor.
There is a list of security advisories affecting VMWare\footnote{\url{http://www.vmware.com/uk/security/advisories/}} though it also includes some other products.
Where do we get version distribution information from?

\printbibliography

\end{document}