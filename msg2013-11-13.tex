\documentclass{beamer}

% Copyright 2004 by Till Tantau <tantau@users.sourceforge.net>.
%
% In principle, this file can be redistributed and/or modified under
% the terms of the GNU Public License, version 2.
%
% However, this file is supposed to be a template to be modified
% for your own needs. For this reason, if you use this file as a
% template and not specifically distribute it as part of a another
% package/program, I grant the extra permission to freely copy and
% modify this file as you see fit and even to delete this copyright
% notice. 


\mode<presentation>
{
  %\usetheme{Warsaw}
  %\usetheme{Goettingen}
  %\usetheme{Montpellier}
  \usetheme{Boadilla}
  % or ...

  \setbeamercovered{transparent}
  % or whatever (possibly just delete it)
}
\beamertemplatenavigationsymbolsempty

%Handout/Notes
%\setbeameroption{show notes} % un-comment to see the notes
%\setbeamertemplate{note page}[plain]
%\usepackage{pgfpages}
%\pgfpagesuselayout{2 on 1}[a4paper,border shrink=5mm]


\usepackage[english]{babel}
% or whatever

\usepackage[latin1]{inputenc}
% or whatever

\usepackage{times}
\usepackage[T1]{fontenc}
% Or whatever. Note that the encoding and the font should match. If T1
% does not look nice, try deleting the line with the fontenc.
\usepackage{tikz}

\title[Security Updates] % (optional, use only with long paper titles)
{Security of deployed Android devices}

\subtitle
{When do Android devices get security updates for root exploits} % (optional)
%\pgfdeclareimage[height=2.9cm]{drt24}{img/drt24}
%\titlegraphic{\pgfuseimage{drt24}}

\author[Daniel Thomas] % (optional, use only with lots of authors)
{Daniel Thomas\\ \vspace{1em} \tiny{(PhD student, 2\textsuperscript{nd} year)} \\ \vspace{1em} \tiny{Supervised by Alastair Beresford}}
\institute{\vspace{-1em}
University of Cambridge
}


\date[2013-11-13] % (optional)
{Mobile Security Reading Group WIP presentation}

\subject{Mobile Security Reading Group WIP presentation}
% This is only inserted into the PDF information catalog. Can be left
% out. 



% If you have a file called "university-logo-filename.xxx", where xxx
% is a graphic format that can be processed by latex or pdflatex,
% resp., then you can add a logo as follows:

 \pgfdeclareimage[height=0.7cm]{university-logo}{university-logo}
 \logo{\pgfuseimage{university-logo}}


% If you wish to uncover everything in a step-wise fashion, uncomment
% the following command: 

%\beamerdefaultoverlayspecification{<+->}

% Maximum of 12 slides including title page.
% 15 minutes talk

\begin{document}

\begin{frame}
  \titlepage
  \note{We collected data with Device Analyzer, collated a list of root equivalent exploits which apps could use and found that a large proportion of devices are vulnerable.}
\end{frame}


\begin{frame}{Device Analyser collects lots of Data}{}
 \begin{itemize}
  \note{}
  \item Android app by Daniel Wagner installed on user devices collecting data with consent.
  \note{\\}
  \item Deployed since May 2011
  \note{\\}
  \item ??? records, ??? years, ??? devices
  \item \url{http://deviceanalyzer.cl.cam.ac.uk/}
 \end{itemize}
 \pgfdeclareimage[height=50px]{dalogo}{figures/dalogo100}
 \begin{center}
  \pgfuseimage{dalogo}
 \end{center}
\end{frame}

\begin{frame}{Device Analyzer collects different kinds of data}{}
% Specific research aim (3+ year time scale)
 \pgfdeclareimage[height=0.38\textheight]{da_main}{figures/da_screenshot_main_screen}
 \pgfdeclareimage[height=0.38\textheight]{da_app_data}{figures/da_screenshot_app_data}
 \pgfdeclareimage[height=0.38\textheight]{da_phone_usage}{figures/da_screenshot_phone_usage}
 \begin{center}
  \pgfuseimage{da_main} \hspace{1em} \pgfuseimage{da_app_data} \hspace{1em} \pgfuseimage{da_phone_usage}
 \end{center}
 \vspace{-1.5em}
 \begin{itemize}
  \item DA also collects information on the OS version information and build number.
  \note{}
 \end{itemize}
    
\end{frame}

\begin{frame}{Root equivalent vulnerabilities are interesting}{}
 \begin{itemize}
  \item Root exploits break out of the sandbox.
  \note{}
  \item They allow applications to do arbitrarily bad things
  \item They are used in ??-??\% of malware
  \item How many devices are vulnerable to these exploits?
 \end{itemize}
\end{frame}

\begin{frame}{Root equivalent vulnerabilities on Android}{}
\small \url{https://wiki.dtg.cl.cam.ac.uk/drt24/dasecurityupdates}\\
All Android vulnerabilities:
 \begin{itemize}
  \item exploid
  \item RageAgainstTheCage zygote
  \item RageAgainstTheCage adb setuid exhaustion
  \item KillingInTheNameOf psneuter ashmem
  \item zergRush
  \item levitator
  \item Gingerbreak
  \item APK signature verification - duplicate file entries
  \item APK signature verification - unsigned shorts
 \end{itemize}

\end{frame}

% Leverage of colleagues
\begin{frame}{Old versions of Android still in use}{Coarse grained API information}
 \pgfdeclareimage[width=0.5\textwidth]{da_norm_api}{figures/da_norm_api}
 \pgfdeclareimage[width=0.5\textwidth]{play_norm_api}{figures/play_norm_api}
 \begin{center}
  \pgfuseimage{da_norm_api}
  \pgfuseimage{play_norm_api}
 \end{center}
Device Analyzer data is representative of the versions of Android actually in use
\end{frame}

\begin{frame}{Old versions of Android still in use}{Detailed OS version}
 \pgfdeclareimage[height=0.87\textheight]{da_norm_os}{figures/da_norm_os}
 \begin{center}
  \vspace{-0.5em}
  \pgfuseimage{da_norm_os}
 \end{center}
\end{frame}

\begin{frame}{Lots of Android devices are exposed to known root equivalent vulnerabilities}{}
 \pgfdeclareimage[height=0.87\textheight]{da_proportioninsecure}{figures/proportioninsecure}
 \begin{center}
  \vspace{-0.5em}
  \pgfuseimage{da_proportioninsecure}
 \end{center}
\end{frame}

\begin{frame}{Several vulnerabilities contribute to this insecurity}{}
 \pgfdeclareimage[height=0.95\textheight]{da_vulnerabilities}{figures/vulnerabilities}
 \begin{center}
  \vspace{-0.5em}
  \pgfuseimage{da_vulnerabilities}
 \end{center}
\end{frame}

\begin{frame}{Devices do get updated}{}
 \pgfdeclareimage[height=0.95\textheight]{da_updates_between_versions}{figures/updates_between_versions}
 \begin{center}
  \vspace{-0.5em}
  \pgfuseimage{da_updates_between_versions}
 \end{center}
\end{frame}

\begin{frame}{Devices do get security updates}{}
 \pgfdeclareimage[height=0.95\textheight]{da_w_security_updates}{figures/w_security_updates}
 \begin{center}
  \vspace{-0.5em}
  \pgfuseimage{da_w_security_updates}
 \end{center}
\end{frame}

\begin{frame}{There are weaknesses in our current approach}{}
 \begin{itemize}
  \item What complementary data should we be using?
  \item Do we have all known root vulnerabilities? (No)
  \item Is our inference that if a device is running a version of Android with a known vulnerability then it is vulnerable true?
  \item Is there anything else we can add to Device Analyzer to improve research into this area?
 \end{itemize}
\end{frame}

\end{document}